\documentclass{article}
\usepackage{proof}
\usepackage{bussproofs}
\usepackage{hyperref}

\begin{document}

\title{Proof Tree Examples}
\author{Nathan Collins}

\maketitle
\section{Proof Setting Packages}
See \url{http://www.logicmatters.net/latex-for-logicians/nd/}.

\section{Using \texttt{proofs.sty}}
Appears to support arbitrarily many premises.

The syntax is
\begin{verbatim}
\infer[<optional rule name>]
  {<conclusion>}
  {<premise 1> & ... & <premise n>}
\end{verbatim}
\[
\infer[^{(2)}]
     {\neg(\phi \land \psi)}
     {\infer[^{(1)}]
        {\bot}
        {(\neg\phi \lor \neg\psi) &
        \infer
            {\bot}
            {\phi &
            \infer[^{(1)}]
            {\neg\phi}{}
            }
        & \bot & \bot & \bot & \bot & \bot & \bot & \bot
        & \Gamma \vdash A & \Gamma \vdash A & \Gamma \vdash A & \Gamma \vdash A & \Gamma \vdash A & \Gamma \vdash A
        }
     }
\]
\section{Using \texttt{bussproofs.sty}}
Proofs are limited to five premises!  Looking at the source, I'm
optimistic that it would not be very hard to create versions up to any
particular premise arity, but it looks making the arity a parameter
would be a major rewrite.

\begin{prooftree}
\AxiomC{$\Gamma, A \vdash B$}
\RightLabel{$\to$ I}
\UnaryInfC{$\Gamma \vdash A \rightarrow B$}
\end{prooftree}

\begin{prooftree}
\AxiomC{$\Gamma \vdash A \to B$}
\AxiomC{$\Gamma \vdash A$}
\RightLabel{$\to$ E}
\BinaryInfC{$\Gamma \vdash B$}
\end{prooftree}

\end{document}